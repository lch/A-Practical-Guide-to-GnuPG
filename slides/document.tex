\documentclass{beamer}


\mode<presentation> {
	\usetheme{Madrid}
	%\setbeamertemplate{footline} % To remove the footer line in all slides uncomment this line
	%\setbeamertemplate{footline}[page number] % To replace the footer line in all slides with a simple slide count uncomment this line
	
	%\setbeamertemplate{navigation symbols}{} % To remove the navigation symbols from the bottom of all slides uncomment this line
}

\usepackage{ctex}
\usepackage{color}
\usepackage{listings}
\usepackage{hyperref}

\title{GnuPG 实践\\ A Practical Guide to GnuPG}
\author{lch)}
\date{\today}

\institute[LZUOSS]{
	\href{mailto:lchopn@gmail.com}{\textit{lchopn@gmail.com}}
}


\begin{document}
	\AtBeginSection{
		\begin{frame}
			\frametitle{Index}
			\tableofcontents[currentsection]
		\end{frame}
	}
	\AtBeginSubsection{
		\begin{frame}
		\frametitle{Index}
			\tableofcontents[currentsubsection]
		\end{frame}
	}
	\maketitle
	\section{什么是 GnuPG ?}
		\subsection{GnuPG 简介}
			\begin{frame}{GnuPG 简介}
				\begin{itemize}
					\item GnuPG 是一个用于加密、数字签名及产生非对称匙对的软件。
					\item GnuPG 是自由软件基金会(Free Software Foundation)的 GNU 计划 (GNU's not Unix)的一部分。
				\end{itemize}
			\end{frame}
			\begin{frame}{从 PGP(Pretty Good Privacy)说起}
			\begin{itemize}[<+->]
				\item 由 Phil R. Zimmermann 在 1991 年开发并免费于网络上发放。
				\item 由于美国的出口限制,PGP 二进制程序、源代码在法律上不允许被传播到美国本土以外。
				\item PGP 的作者将源代码由 MIT Press 出版并印刷成了一本书《PGP Source and Internals》,由于书籍的出口受到美国宪法第一修正案的保护,任何人都可以以 \$60 购买这本书,并将源代码通过 OCR 扫描并录入电脑,并使用编译器编译为二进制文件。
				\item PGP 仍然是商业软件,属于 PGP Inc.(Zimmermann 作为公司 CTO 与 Chairman),后并入 Network Associate (NAI,McAfee)现在已经被 Symantec 收购。
				\item 在 1997 年 Zimmermann 与 IETF 共同制订 OpenPGP 标准(RFC1991 等)。
			\end{itemize}
			\end{frame}
			\begin{frame}{OpenPGP 与 GnuPG}
				\begin{itemize}
					\item GnuPG (GNU Privacy Guard) 基于 OpenPGP 标准开发,是 OpenPGP 标准的一个实现。
					\item GnuPG 最初由德国人 Werner Koch 开发。
					\item GnuPG 以 GNU Public License 发布,是自由软件。
				\end{itemize}
			\end{frame}
		\subsection{对称加密与非对称加密}
			\begin{frame}{对称加密与非对称加密}
				\begin{block}{对称加密}
					使用同一个密钥加密和解密(DES、AES)
				\end{block}
				\begin{block}{非对称加密}
					使用不同的密钥加密解密(RSA、ECC)
				\end{block}
			\end{frame}
		\subsection{获取 GnuPG}
			\begin{frame}{获取 GnuPG}
				\begin{block}{GNU/Linux}
					大多数 GNU/Linux 发行版自带 GnuPG。
				\end{block}
				\begin{block}{Windows}
					Gpg4win \href{https://www.gpg4win.org/download.html}{https://www.gpg4win.org/download.html}
				\end{block}
				\begin{block}{macOS}
					GPG Suite \href{https://gpgtools.org/}{https://gpgtools.org/}
				\end{block}
				\begin{block}{Android}
					OpenKeychain \href{https://www.openkeychain.org/}{https://www.openkeychain.org/}
				\end{block}
			\end{frame}
	\section{GnuPG 最佳实践}
		\subsection{Do it in right way!}
			\begin{frame}[fragile]{使用 GnuPG 2}
				\begin{itemize}[<+->]
					\item 请使用 GnuPG 2。
					\item GnuPG 1 与 2 的区别
					\begin{itemize}
						\item GnuPG 2 提供新的加密算法。
						\item GnuPG 1 作为独立版拥有最小依赖,提供给嵌入式系统或服务器。
						\item GnuPG 1 为了兼容可能还提供一部分对 PGP 2.6 的支持。 
					\end{itemize}
				\end{itemize}
				\begin{block}{From manpage}\texttt{This is the standalone verion of gpg. For desktop use you should consider using gpg2 from GnuPG-2 Package.}
					\\
					\texttt{... GnuPG 1.x, which is might be better suited for server and embedded platforms...}
				\end{block}
\end{frame}
		\begin{frame}{不要使用简单生成}
			最新版本的 GnuPG 在使用 \texttt{gpg2 --gen-key} 命令的时候会默认产生 RSA-2048 bit 的一个主密钥和一个子密钥,并且主密钥默认开启签名和认证功能,子密钥默认开启加密功能,并且采用 SHA1 作为摘要算法。这一默认设置存在问题,故要尽量避免使用简单的 \texttt{--gen-key} 参数生成密钥。
		\end{frame}
		\begin{frame}{编辑配置文件}
			在开始之前,你需要编辑\texttt{\$HOME/.gnupg/gpg.conf}、\texttt{\$HOME/.gnupg/gpg-agent.conf},以提高安全性!
		\end{frame}
		\begin{frame}{生成密钥对}
			\begin{itemize}
				\item 生成认证用主 Key
				\item 增加签名用子 Key
				\item 增加加密用子 Key
				\item 增加认证用子 Key
			\end{itemize}
		\end{frame}
		\begin{frame}{我该使用多少位的 RSA 密钥?}
		\begin{itemize}
			\item 	根据官方文档,2048 bit 的密钥理论上可以安全使用至 2020 年,并且在默认创建的密钥中使用了 2048bit 的密钥,也没有推荐或者反对使用更长的密钥。
			\item 尽管如此,我们仍然强烈建议在使用 RSA 密钥的时候选择更长的密钥,强烈推荐使用 4096bit 的密钥。
		\end{itemize}
	\end{frame}
		\subsection{密钥使用与管理}
			\begin{frame}{加密/解密}
			\begin{block}{加密}
				\texttt{gpg2 --encrypt --sign --recipient john@example.com --armor --output /tmp/very-secret-message.gpg /tmp/clear-text.txt}
			\end{block}
			\begin{block}{解密}
				\texttt{gpg2 --decrypt --output /tmp/clear-text.txt /tmp/very-secret-message.gpg}
			\end{block}
		\end{frame}
		\begin{frame}{发布与签名}
			\begin{block}{发布你的公钥}
				\texttt{gpg2 --send-keys 0x09F33B41FEDB1EE9}
			\end{block}
			\begin{block}{签名一个 Key}
				\textbf{验证签名信息!!!}
				\begin{itemize}[<+->]
					\item 签名方
					\texttt{gpg2 --list-public-keys --with-icao-spelling 0x09F33B41FEDB1EE9}
					\item 被签名方 
					\texttt{gpg2 --list-secret-keys --with-icao-spelling 0x09F33B41FEDB1EE9}
					\item 签名
					\texttt{gpg2 --sign-key john@example.com}
					\item 发送签名信息
					\texttt{gpg2 --send-keys john@example.com}
					\item 验证用签名
					\texttt{gpg2 --lsign-key john@example.com}
				\end{itemize}
			\end{block}
		\end{frame}
		\begin{frame}{密钥管理}
		\begin{block}{生成吊销证书}
			\texttt{gpg2 --output gpg-0x09F33B41FEDB1EE9.asc --gen-revoke 0x09F33B41FEDB1EE9} \\
			\textbf{务必保证}这一吊销证书保存在一个\textbf{安全}的地方,这一文件将能\textbf{直接吊销}你的密钥。如果可能的话,将其手写在纸上(一些打印系统甚至会记录你打印的东西),并保存在一个私密的地方。
		\end{block}
	\end{frame}
	\begin{frame}{密钥存储}
		\begin{itemize}[<+->]
			\item 在线存储
			\begin{itemize}
				\item 主密钥保存在本地,要求你必须信任你在本地安装的所有软件。(非常不安全)
			\end{itemize}
			\item 离线存储
				\begin{itemize}
					\item 智能卡(YubiKey、GnuK 等等)
					\item USB 闪存、Tails
					\item 一台独立的、离线的电脑
				\end{itemize}
		\end{itemize}
\end{frame}
	\begin{frame}{删除主密钥的私钥}
		\begin{itemize}[<+->]
			\item 首先备份整个 \texttt{\~{}/.gnupg} 文件夹到 U 盘上。
			\item 如果你的 GnuPG 版本 > 2.1,那么只需要删除\\ \texttt{\$HOME/.gnupg/private-keys-v1.d/KEYGRIP.key}
			\item 如果你的 GnuPG 版本 < 2.1
			\begin{itemize}
				\item 你需要导出所有子密钥\\
				\texttt{gpg2 --output secret\_subkeys --export-secret-subkeys YOURMASTERKEYID}
				\item 删除主密钥私钥 \\
				\texttt{gpg --delete-secret-keys YOURMASTERKEYID}
				\item 导入子密钥私钥
				\texttt{gpg --import secret\_subkeys}
				\item 删除导出文件 \texttt{secret\_subkeys}
			\end{itemize}
		\end{itemize}
	\end{frame}
	\begin{frame}
		\begin{itemize}
			\item 此时可以检查 \texttt{gpg2 --list-keys} ,私钥前的 \texttt{sec} 已经变成 \texttt{sec\#}。表明主密钥私钥并不真正保存在此处。
			\item 此时还可以使用 \texttt{gpg --edit-key YOURMASTERKEYID passwd} 修改密码,防止日常使用中密码泄漏影响主密钥的安全。
		\end{itemize}
\end{frame}
	\section{总结}
		\begin{frame}{总结}
		\begin{itemize}
			\item 务必保证私钥安全
			\item 再小心都不为过
			\item Take responsibility to your action.
		\end{itemize}
	\end{frame}
	
	\begin{frame}{参考文献与延伸阅读}
		\begin{itemize}
			\item \href{https://phab.enlightenment.org/w/gnupg/}{https://phab.enlightenment.org/w/gnupg/}
			\item \href{https://wiki.debian.org/Subkeys}{Debian Wiki: Subkeys}
			\item \href{https://www.gnupg.org/gph/de/manual/r1023.html}{GnuPG manpage}
			\item \href{https://www.gnupg.org/gph/en/manual/book1.html}{The GNU Privacy Handbook}
			\item \href{https://begriffs.com/posts/2016-11-05-advanced-intro-gnupg.html}{An Advanced Intro to GnuPG}
			\item \href{https://en.wikipedia.org/wiki/Public-key_cryptography}{Public-key cryptography - Wikipedia}
			\item \href{https://en.wikipedia.org/wiki/Symmetric-key_algorithm}{Symmetric-key algorithm - Wikipedia}
			\item \href{https://en.wikipedia.org/wiki/Pretty_Good_Privacy}{Pretty Good Privacy - Wikipedia}
			\item \href{https://en.wikipedia.org/wiki/GNU_Privacy_Guard}{GNU Privacy Guard - Wikipedia}
			\item Applied Cryptography: Protocols, Algorithms, and Source Code in C, Bruce Schneier, Wiley
		\end{itemize}
	\end{frame}

\begin{frame}
~ \\ ~ \\ ~ \\
\Large{Thank you for your attention!} \\ ~ \\ ~ \\
\small{This work was licensed under CC-BY-NC-SA 4.0 , source code was licenced under GNU GPLv3.}
\end{frame}
\end{document}